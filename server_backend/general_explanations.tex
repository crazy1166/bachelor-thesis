\section{General explanations}
%
We will here describe the general working of the backend application. % This sentece may be verbosity.

% I recommend that below sentence is divided into multiple sentences.
The application is launched asynchronously from the frontend application, 
therefore, when starting up the backend application first sends a message to 
the frontend to let it know that the application is now started and ready to 
receive and transfer messages. This is a single-way communication and the 
application does not wait for any answer from the frontend.

The second step when starting up the backend is to start listening on the 
port the X clients will be connecting to. The hostname 
to listen is statically set using a configuration file, however as the port 
to listen to changes depending on the current screen number, the port number to 
use to listen for incoming connection is sent as a command-line argument of the 
application. Typically, the first application to start will be given the 
screen number $0$, and will therefore be listening on the port $6000$ 
(see~\ref{ssec:connection-setup}).
% What port number is assigned to Second and Third request?

Finally, the application reads the initialization file of the current user 
to start launching applications (eg. a window manager or desktop environment). 
This file should be placed in the user home directory, and is called 
\lstinline{.scalaxsrc}, however we plan to use the file \lstinline{.xinitrc} to 
try to imitate the behavior of \lstinline{startx}
\footnote{\href{http://linux.die.net/man/1/startx}{\lstinline{startx}} is a script used to start the X Window System.}
or SLiM
\footnote{\href{http://slim.berlios.de/}{SLiM (Simple Login Manager)} is a graphical login manager for X.}.

Once the application have finished initializing, it should received the first requests 
from X clients launched when reading the initialization file.

When a request is received, the information given by the request, if any, is cached 
by the application, and the request is handled locally when possible, otherwise 
transferred to the frontend application. We will discuss about this further in the next 
section.

To handle requests sent by the X clients, the application first parses the requests and 
transforms it in a normal Scala object. It then checks if the requests needs to be handled 
locally, if the content of the requests needs to be cached, and finally if the request 
needs to be transferred to the frontend and sends it to the frontend through a TCP connection 
when needed.
