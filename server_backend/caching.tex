\section{Caching}
%
As said in the previous section, an important responsibility of the backend application is to
cache data received by both the X clients and the frontend application, and to use this cache 
efficiently to try to reduce the most possible the requests and replies transfers to the 
frontend application.

The caching system has not yet been implemented in the system when writing this paper, but the 
strategy to use is straightforward and should not be too difficult to implement.

The first step is to check if the requests content needs to be cached or not. 
As the requests are implemented as Scala case classes, we could easily add a trait 
to the requests that need to be cached. A possible example of this implementation is shown 
in listing \ref{list:caching}.

\begin{lstlisting}[basicstyle=\footnotesize,caption=Possible implementation of requests needing cache,language=scala,label=list:caching]
trait NeedsCaching { // define trait for caching
  def cacheRequest(): Unit
}

// mixin traits to class needing to be cached
case class CreateWindowRequest (
 ...
) extends Request 
  with NeedsTransfer // needs to be transfered to the frontend
  with NeedsCaching { // needs to be cached 
  ...
  def cacheRequest() {
    // do caching logic
  }
}

// handle requests needing caching
request match {
  case r: NeedsCaching => r.cacheRequest()
  ...
}
\end{lstlisting}
Of course, the listing above is quite minimalist, but this should be enough to check if requests need 
caching and handle them. 

The content of the \lstinline{cacheRequest} method being too different 
for each request, we will not enter in the possible implementation details here, but at least 
for the different resources, the idea would be to save them as an appropriate model instance 
after checking that the resource id is valid.

Using this method, the errors of access to unavailable could be done in the backend application 
without any need to transfer the request further.
