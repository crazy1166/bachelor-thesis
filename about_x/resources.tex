\section{Resources}
The protocol uses a number of different resources which we will shortly 
introduce here.
\begin{description}
\item[Drawable] A drawable is an abstract entity used to draw. It can be a window or a pixmap.
\item[Window] There are two types of window: top-level window and subwindows. 
  A top-level window is the main container for an application and 
  usually contains all the other components of the application.
  A subwindow is a window contained in the top-level window of an application 
  and can be used for anything, from the titlebar to a button in the application.
\item[Pixmap] A pixmap is a region used to draw, but on opposite to a window, 
  it is not shown until explicitly requested. The content of a pixmap 
  or part of it can be displayed on a window.
\item[Graphic context] A graphic context is a structure containing basic graphic 
  information to apply to a given request, for example the background and 
  foreground colors, or a transformation to apply to the shape to draw.
\footnotetext{Diagram from Wikipedia \href{http://en.wikipedia.org/wiki/X_Window_System_core_protocol}{X Window System Core Protocol}}
\end{description}
