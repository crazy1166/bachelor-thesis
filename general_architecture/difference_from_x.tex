\section{Differences with a normal X server}
In this section, we will discuss between the main differences of this project 
with a normal X server.
\subsection{Divided server}
In a normal X client/server architecture, the client connects to the server, 
and communicates directly with it. The X server needs to have control over the 
keyboard, mouse, screen, and other hardware used to display graphics, and handle events.

This system is very different as first of all, there is not really \emph{an} X server.
Given the system requirements, the application that communicates with the client and the 
application that listens to the events cannot be the same; handling all the TCP 
directly in the web browser is not realistic. There is therefore a need 
to communicate with clients, and make communication possible with the application 
acting as the X server frontend.
%
\subsection{Communication through websockets}
As a consequence of the above difference, another difference is that to 
implement this system, we need not only to communicate with X clients using 
a bytestream such as TCP sockets, but we also need to use some communication protocol to 
communicate with a web browser. The chosen protocol to achieve this is 
websockets, we will be discussing further about this decision later in 
this paper.
%
\subsection{Integrated login manager}
This project goes further from a normal X server in the sense that the 
login manager is integrated and therefore does not depends on the 
X Display Manager Control Protocol.