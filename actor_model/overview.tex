\section{Overview}
%
First of all, we are going to briefly explain why and how can this model be useful, 
and then explain how it works.
%
\subsection{Motivations}
%
The first motivation for the actor model is too make multi-threading simpler and safer.
To give an overview of the problem to be solved, in listing \ref{list:pthread-example}
we are going to take a short example written in C++ using the pthread 
library for threading.
%
\begin{lstlisting}[basicstyle=\footnotesize,caption=Threading example in C++,language=C++,label=list:pthread-example]
bool resource_manager::get_resource(int id, resource& r)
{
  pthread_mutex_lock(&this->_resource_lock);
  if(!this->has_resource(id)) {
    return false;
  }
  r = this->_resource[id];
  pthread_mutex_unlock(&this->_resource_lock);
  return true;
}

// in some other method
if(!manager->get_resource(wanted_id, wanted_resource)) {
  // if wanted id not available get fallback instead
  manager->get_resource(fallback_id, wanted_resource);
}
\end{lstlisting}
In the above example, the application will deadlock as soon as the \lstinline{wanted_id}
is not available in the resource manager, as the lock is not released in that case.

Of course, in a simple example like this, any programmer having a little experience 
with threads would notice immediately that the lock is not released and it would 
probably would not take too long to debug anyway. However, everything is not always 
so simple, and as timing plays an important role when working with threads, 
deadlocks can be very difficult to debug.