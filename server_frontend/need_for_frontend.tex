\section{Need for the frontend}
Given the above explanations about the frontend application, one could 
think that the communication could be directly be done between the 
server backend and the web browser, which is partly true. We will therefore
briefly discuss about the need for this application.
%
\subsection{Security concerns}
The most important reason of this application to be is basic security concerns. 
The main point being
\begin{itemize}
\item If the backend application and the web browser were communicating directly, 
  as what user should the backend application be run?   
\end{itemize}
which is a question with probably not any answer. If only a single user was to 
use the system, the application could be run as this user, which depending on the 
port the application is running on, could be done without having to change any privilege.
However, this is no longer a valid approach when several users want to use the application 
at the same time. In this case, the application could be running as a special user 
created specially for that purpose, but in this case, for the user to be able to access 
their home, this user should be allowed to access it too, which clearly is not reasonable.

The best solution to solve this problem was therefore to create a kind of dispatcher application, 
which act as a frontend, and starts the backend application in a new process as the given user 
for each connection. Using this method, the only requirement for the frontend is to have a
way to check the password authentication and to start a new process as this user. We will 
discuss about this in the next section.
%
\subsection{Distributed system}
Another motivation for adding a frontend is that by letting it act as a dispatcher 
of the backend application, adapting the system for it to run on several remote computers
should be greatly simplified. 

With the current system design, the frontend application and the backend application have no need 
to be running on the same machine, and the frontend could therefore act not only as a dispatcher 
to launch different process on the active machine, but also as a dispatcher to run the processes 
on different machines, on the same local area network, for example. As the communications 
between the frontend and backend are already done through TCP, this part of this 
application would work without any change, except for the hostname configuration.

% Below is my comment.
% I think that the web application and the backend application can connect directly,
% after login process.
% If the login informaion is valid, the login manager tells the port number, 
% which the backend application is waiting, to the web application.
% The web application can establish the connection to the backend application using the port.
